%ADASS_PROCEEDINGS_FORM%%%%%%%%%%%%%%%%%%%%%%%%%%%%%%%%%
%
% SAMPLE2.TEX -- ADASS XVII (2007)-- ADASS Conference Proceedings sample
% paper with complicated markup. Based on ADASS XI (01) version.
%
% This is a comprehensive example, meaning that we have made use of each
% of the capabilities of the LaTeX + the ASPCONF macro package that we think
% you may need to use. 
%
% Much of the input will be enclosed by braces (i.e., { }).  The
% percent sign, "%", denotes the start of a comment; text after it
% will be ignored by LaTeX.  You might also notice in some of the
% examples below the use of "\ " after a period; this prevents LaTeX
% from interpreting the period as the end of a sentence and putting
% extra space after it.   
% 
% You should check your paper by processing it with LaTeX.  For
% details about how to run LaTeX as well as how to print out the User
% Guide, consult the README file.  
%
%%%%%%%%%%%%%%%%%%%%%%%%%%%%%%%%%%%%%%%%%%%%%%%%%%
% 
\documentclass[11pt,twoside]{article}  % Leave intact
\usepackage{asp2006}
\usepackage{adassconf}

\begin{document}   % Leave intact

%-----------------------------------------------------------------------
%			    Paper ID Code
%-----------------------------------------------------------------------
% Enter the proper paper identification code.  The ID code for your paper 
% is the session number associated with your presentation as published 
% in the official conference proceedings.  You can find this number by 
% locating your abstract in the printed proceedings that you received 
% at the meeting, or on-line at the conference web site.
%
% This identifier will not appear in your paper; however, it allows different
% papers in the proceedings to cross-reference each other.  Note that
% you should only have one \paperID, and it should not include a
% trailing period.

\paperID{O4.1}

%-----------------------------------------------------------------------
%		            Paper Title 
%-----------------------------------------------------------------------
% Enter the title of the paper.
%
% EXAMPLE: \title{A Breakthrough in Astronomical Software Development}

\title{Youpi, a Web-based Astronomical Image Processing Pipeline}
       
%-----------------------------------------------------------------------
%          Short Title & Author list for page headers
%-----------------------------------------------------------------------
% Please supply the author list and the title (abbreviated if necessary) as 
% arguments to \markboth.
%
% The author last names for the page header must appear in one of 
% these formats:
%
% EXAMPLES:
%     LASTNAME
%     LASTNAME1 and LASTNAME2
%     LASTNAME1, LASTNAME2, and LASTNAME3
%     LASTNAME et al.
%
% Use the "et al." form in the case of four or more authors.
%
% If the title is too long to fit in the header, shorten it: 
%
% EXAMPLE: change
%    Rapid Development for Distributed Computing, with Implications for the Virtual Observatory
% to:
%    Rapid Development for Distributed Computing

%\markboth{Djorgovski, King, and Biemesderfer}{Collapsed Cores in Globular Clusters}
\markboth{Monnerville and Semah}{Youpi, a Web-based Astronomical Image Processing Pipeline}

%-----------------------------------------------------------------------
%		          Authors of Paper
%-----------------------------------------------------------------------
% Enter the authors followed by their affiliations.  The \author and
% \affil commands may appear multiple times as necessary.  List each
% author by giving the first name or initials first followed by the
% last name. Do not include street addresses and postal codes, but 
% do include the country name or abbreviation. 
%
% If the list of authors is lengthy and there are several institutional 
% affiliations, you can save space by using the \altaffilmark and \altaffiltext 
% commands in place of the \affil command.

%\author{S.\ Djorgovski\altaffilmark{1,2}, Ivan R.\ King}
%\affil{Astronomy Department, University of California, Berkeley, CA, USA}

\author{M.\ Monnerville}
\affil{Terapix, Institut d'Astrophysique de Paris, University of Pierre et Marie Curie, UMR 7095, Paris, France}

\author{G.\ Semah}
\affil{Terapix, Institut d'Astrophysique de Paris, CNRS, UMR 7095, Paris, France}

% Notice that some of these authors have alternate affiliations, which
% are identified by the \altaffilmark after each name.  The actual alternate
% affiliation information is typeset in footnotes at the bottom of the
% first page, and the text itself is specified in \altaffiltext commands.
% There is a separate \altaffiltext for each alternate affiliation
% indicated above.

%\altaffiltext{1}{Visiting Astronomer, Cerro Tololo Inter-American Observatory. 
%CTIO is operated by AURA, Inc.\ under cooperative agreement with the National
%Science Foundation} 
%\altaffiltext{2}{Society of Fellows, Harvard University} 
%\altaffiltext{3}{Patron, Alonso's Bar and Grill}

%-----------------------------------------------------------------------
%			 Contact Information
%-----------------------------------------------------------------------
% This information will not appear in the paper but will be used by
% the editors in case you need to be contacted concerning your
% submission.  Enter your name as the contact along with your email
% address.

\contact{Mathias Monnerville}
\email{monnerville@iap.fr}

%-----------------------------------------------------------------------
%		      Author Index Specification
%-----------------------------------------------------------------------
% Specify how each author name should appear in the author index.  The 
% \paindex{ } should be used to indicate the primary author, and the
% \aindex for all other co-authors.  You MUST use the following syntax: 
%
%    \aindex{LASTNAME, F.~M.}
% 
% where F is the first initial and M is the second initial (if used). Please 
% ensure that there are no extraneous spaces anywhere within the command 
% argument. This guarantees that authors that appear in multiple papers
% will appear only once in the author index. Authors must be listed in the order
% of the \paindex and \aindex commmands.

%\paindex{Djorgovski, S.}
%\aindex{King, I.~R.}
%\aindex{Biemesderfer, C.~D.}

\paindex{Monnerville, M.}
\aindex{Semah, G.}

%-----------------------------------------------------------------------
%			Subject Index keywords
%-----------------------------------------------------------------------
% Enter up to 6 keywords that are relevant to the topic of your paper.  These 
% will NOT be printed as part of your paper; however, they will guide the creation 
% of the subject index for the proceedings.  Please use entries from the
% standard list where possible, which can be found in the index for the 
% ADASS XVI proceedings. Separate topics from sub-topics with an exclamation 
% point (!). 

%\keywords{astronomy!globular clusters}
\keywords{astronomy!pipeline clusters}

% We reset the footnote counter for the hyperlink since it does not
% appear to recognize the previous 3 footnotes generated from the
% altaffilmarks.  

\setcounter{footnote}{3}

%-----------------------------------------------------------------------
%			       Abstract
%-----------------------------------------------------------------------
% Type abstract in the space below.  Consult the User Guide and Latex
% Information file for a list of supported macros (e.g. for typesetting 
% special symbols). Do not leave a blank line between \begin{abstract} 
% and the start of your text.

%\begin{abstract}          % Leave intact
%This is a preliminary report on surface photometry of the major
%fraction of known globular clusters, to see which of them show the
%signs of a collapsed core.  We also show off the results of some
%recreational mathematics.
%\end{abstract}

\begin{abstract} 
Youpi stands for ``YOUpi is your processing PIpeline''. It is a modern, 
easy to use yet powerful web application providing high level functionalities 
to perform data reduction on scientific FITS images. Built on top of various 
open source reduction tools released to the community by TERAPIX, Youpi can 
help organize your data, manage your processing jobs on a computer cluster 
in real time (using Condor) and facilitate teamwork by allowing fine-grain 
sharing of results and data.
\end{abstract}

%-----------------------------------------------------------------------
%			      Main Body
%-----------------------------------------------------------------------
% Place the text for the main body of the paper here.  You should use
% the \section command to label the various sections; use of
% \subsection is optional.  Significant words in section titles should
% be capitalized.  Sections and subsections will be numbered
% automatically. 

\section{Introduction}

Dealing with data reduction can quickly become a tedious task especially 
from an organizational point of view: one have to remember every input data 
paths involved in the whole processing, have to deal with many obscur command 
lines or scripts. Youpi has been designed to reduce the hassle of dealing with 
obscur command line operations.

%A focal problem today in the dynamics of globular clusters is core
%collapse.  It has been predicted by theory for decades (H\`enon 1961;
%%Lynden-Bell \& Wood 1968; 
%Spitzer 1985; Roberts 2003), but observation has
%been less alert to the phenomenon. For many years the central brightness
%peak in M15 (King 1975) seemed a unique anomaly.
%Then Auri\`ere (1982) suggested a central peak in NGC 6397, and a limited
%photographic survey of ours (Djorgovski \& King 1984) found three more
%cases, including NGC 6624, whose sharp center had often been remarked on.

\section{Observations}

% The command \htmladdnormallinkfoot puts the link as a
% footnote in the printed paper.  The command \htmladdnormallink with
% the same arguments will ignore the link in the printed copy.

All our observations were short direct exposures with
\htmladdnormallinkfoot{CCD's}{http://www.noao.edu/}.

\begin{enumerate}
\item At Lick Observatory we used a TI 500$\times$500 chip 
and a GEC 575$\times$385, on the 1-m Nickel reflector.  The only
filter available at Lick was red.
\item At CTIO we used a GEC 575$\times$385, with
$B, V,$ and $R$ filters, and an RCA 512$\times$320, with $U, B, V, R,$
and $I$ filters, on the 1.5-m reflector. In the CTIO observations we
tried to concentrate on the shortest practicable wavelengths; but
faintness, reddening, and poor short-wavelength sensitivity often kept
us from observing in $U$ or even in $B$.
\end{enumerate}

All four cameras had scales of the order of 0\farcs4 pixel$^{-1}$, and our
field sizes were around 3\arcmin.
%
The CCD images are unfortunately not always suitable, for very poor
clusters or for clusters with large cores.  Since the latter are
easily studied by other means, we augmented our own CCD profiles by
collecting from the literature a number of star-count profiles (King
et al.\ 1968; Peterson 1976; Ortolani et al.\ 1985) as well as
photoelectric profiles (King 1966) and electronographic profiles (Kron,
Hewitt, \& Wasserman 1984).  In a few cases we judged normality by eye estimates on
one of the Sky Surveys.

% In this section, we see the use of the \subsection command to set off
% an independent subsection.  We only have one in this example; in 
% general there should be more than one.

\section{Helicity Amplitudes}

It has been realized that helicity amplitudes provide a convenient
means for Feynman-diagram evaluations.  These amplitude-level
techniques are particularly convenient for calculations involving many
Feynman diagrams, where the usual trace techniques for the amplitude
squared becomes unwieldy.  Our calculations use the helicity
techniques developed by other authors (Hagiwara \& Zeppenfeld 1986);
we briefly summarize below.

\subsection{Formalism} \label{hairymath}

% We show the use of several of the displayed math environments
% described in the User Guide, and you get a healthy dose of
% mathematical typesetting examples.  Also, observe the use of the LaTeX
% \label command after the \subsection to give a symbolic tag to the
% subsection for cross-referencing in a \ref command.  LaTeX
% automatically numbers the sections, equations, tables, etc. as it
% goes, so in general you don't know what number something is going to
% have.  We'll refer to the "hairymath" section a little later.

A tree-level amplitude in $e^+e^-$ collisions can be expressed in
terms of fermion strings of the form
\begin{equation}
\bar v(p_2,\sigma_2)P_{-\tau}\not\!a_1\not\!a_2\cdots
\not\!a_nu(p_1,\sigma_1)\;,
\end{equation}
where $p$ and $\sigma$ label the initial $e^{\pm}$ four-momenta and
helicities $(\sigma = \pm 1)$, $\not\!a_i=a^\mu_i\gamma_\mu$, and
$P_\tau=\frac{1}{2}(1+\tau\gamma_5)$ is a chirality projection
operator $(\tau = \pm1)$.  The $a^\mu_i$ may be formed from particle
four-momenta, gauge-boson polarization vectors or fermion strings with
an uncontracted Lorentz index associated with final-state fermions.

\subsubsection{Weyl spinors}

The Weyl spinors are given in terms of helicity eigenstates
$\chi_\lambda(p)$ with $\lambda=\pm1$ by
\begin{eqnarray}
u(p,\lambda)_\pm & = & (E\pm\lambda|{\bf p}|)^{1/2}\chi_\lambda(p)\;,
\nonumber \\ & & \\
v(p,\lambda)_\pm & = & \pm\lambda(E\mp\lambda|{\bf p}|)^{1/2}\chi
_{-\lambda}(p) \nonumber
\end{eqnarray}

% In these sections, we see some additional math-related markup, and we
% have references to one of the tables (occurs later in the document)
% and the "hairymath" section immediately preceding this one.
%
% In the second paragraph, note the use of in-text math ($stuff$) including
% a couple of the miscellaneous symbol commands defined in the macro package.
%
% This is the last section of the paper, so there is an \acknowledgments
% section at the end of the main body.

\section{Floating Material and So Forth}

Consider a task that computes profile parameters for a modified
Lorentzian of the form
\begin{equation}
I = \frac{1}{1 + d_{1}^{P (1 + d_{2} )}}
\end{equation}
where
\begin{displaymath}
d_{1} = \sqrt{ \left( \begin{array}{c} \frac{x_{1}}{R_{maj}} 
\end{array} \right) ^{2} + 
\left( \begin{array}{c} \frac{y_{1}}{R_{min}} \end{array} \right) ^{2} }
\end{displaymath}
\begin{displaymath}
d_{2} = \sqrt{ \left( \begin{array}{c} \frac{x_{1}}{P R_{maj}}
\end{array} \right) ^{2} + 
\left( \begin{array}{c} \frac{y_{1}}{P R_{min}} \end{array} \right) ^{2} }
\end{displaymath}
\[x_{1} = (x - x_{0}) \cos \Theta + (y - y_{0}) \sin \Theta \]
\[y_{1} = -(x - x_{0}) \sin \Theta + (y - y_{0}) \cos \Theta \]

In these expressions $x_{0}$,$y_{0}$ is the star center, and $\Theta$
is the angle with the $x$ axis.  Results of this task are shown in
Table~\ref{O4.1-tbl-1}.  It is not clear how these sorts of analyses may
affect determination of $M_{\sun}$ and $M_{\earth}$, but the
assumption is that the alternate results should be less than 90\deg\
out of phase with previous values.

\begin{itemize}
\item Note that enumerated lists work as expected, with
proper margins and indentation. The list at the
beginning of the paper is enumerated.
\item Itemized lists also work as expected, with proper
margins and indentation.  This list is itemized.
\end{itemize}

% Tables can be generated using the deluxetable environment, the same
% environment used in the AASTeX macro package.  Here \scriptsize
% was used to force the table to fit on the printed page.
%
% Column headings are specified within a \tablehead command, using the
% \colhead command to specify each column heading (or the LaTex command
% \multicolumn may be used).  The data is entered between the \startdata
% and \enddata commands.  Use the & as a column delimiter and end each
% data line with \nl.  For details see the User Guide.

\begin{deluxetable}{crrrrrrrrrrr}
\scriptsize
\tablecaption{Terribly relevant tabular information. \label{O4.1-tbl-1}}
\tablehead{
\colhead{Star} & \colhead{Height} & \colhead{$d_{x}$}  &   \colhead{$d_{y}$} & 
\colhead{$n$}  & \colhead{$\chi^2$} & \colhead{$R_{min}$} &
\multicolumn{1}{c}{$P$\tablenotemark{a}} & \colhead{$P R_{maj}$} & 
\colhead{$P R_{min}$} 
} 
\startdata
1 &33472.5 &$-$0.1 &0.4    &53 &27.4  &1.940 &3.900 &68.3  &116.2 \nl
2 &27802.4 &$-$0.3 &$-$0.2 &60 &3.7   &1.510 &2.156 &6.8   &7.5   \nl
3 &29210.6 &0.9    &0.3    &60 &3.4   &1.551 &2.159 &6.7   &7.3   \nl
4 & 9607.4 &$-$0.4 &$-$0.4 &60 &1.4   &1.574 &2.343 &8.0   &8.9   \nl
5 &31638.6 &1.6    &0.1    &39 &315.2 &3.075 &7.488 &92.1  &25.3  \nl
\enddata
% Text for table footnotes must follow the tabular environment but must
% be inside the table environment.  Note that it is OK to put \ref's
% in \tablenotetext.
 
\tablenotetext{a}{Sample footnote for Table~\ref{O4.1-tbl-1}}
\tablecomments{Table end notes apply to the entire table}
\end{deluxetable}

% In the figure environment the \caption command should contain only the 
% caption text.  The "Figure N." identification is generated by the
% \caption command on its own.
%
% In this example we insert the \epscale command before the \plotone
% command to override the default scaling of the figure. 
%
% You cannot use footnotes within figures.

This paragraph has been placed here to illustrate the
relationship between floating elements and running text.  A sample
figure appears in Figure~\ref{O4.1-fig-1}.
\begin{figure}[t]
\epsscale{0.70}
% \plotone{figure.eps}
\caption{A particularly ghostly figure.} \label{O4.1-fig-1}
\end{figure}
We insert the figure in the text file immediately after the sentence
in which it is called out; this happens to be in the middle of a
paragraph.  Notice that this paragraph continues on after the 
\texttt{figure} environment is closed. The typeset text of the paragraph will
not be broken by the figure; rather, the figure will be floated to
a place (top or bottom of a page, possibly after the
paragraph) close to the textual material in which it is embedded in
the source file.
 
% The same tabular data presented above in the deluxetable environment is
% aligned below within the "tabular" environment. Observe
% that our tabular environment is embedded within a "center" environment,
% which is in turn inside a "table" environment.
%
% We need the table environment for autonumbering and caption generation,
% which is why it is not enough to have a centered tabular.
%
% Within the tabular environment, please note that we use no vertical
% rules, and the horizontal rules are inserted with \tableline (*not* \hline).
% Note that a couple of the column headings require special annotation, i.e.,
% footnotes for tables.  They are marked and tagged with \tablenotemark.
% \tablenotemarks could be placed on individual data entries as well,
% but try not to go berserk doing this.
%
% It is necessary to \label tables and figures *after* the \caption has been
% specified because the table/figure number is generated by \caption, not
% by \begin{whatever}.

We repeat Table~\ref{O4.1-tbl-1} which was formatted using the 
\texttt{deluxetable} environment, but this time we use the \LaTeX\ table
environment (see Table \ref{O4.1-tbl-2}).  Either table environment
is acceptable, but the \texttt{deluxetable} environment is preferred.
%
\begin{table}
\caption{Terribly relevant tabular information.} \label{O4.1-tbl-2}
\begin{center}\scriptsize
\begin{tabular}{crrrrrrrrrr}
\tableline\tableline
Star & Height & $d_{x}$ & $d_{y}$ & $n$ & $\chi^2$ & $R_{min}$ &
\multicolumn{1}{c}{$P$\tablenotemark{a}} & $P R_{maj}$ & $P R_{min}$ & 
\multicolumn{1}{c}{$\Theta$\tablenotemark{b}}\\
\tableline
1 &33472.5 &$-$0.1 &0.4    &53 &27.4  &1.940 &3.900 &68.3  &116.2 &$-$27.639 \nl
2 &27802.4 &$-$0.3 &$-$0.2 &60 &3.7   &1.510 &2.156 &6.8   &7.5   &$-$26.764 \nl
3 &29210.6 &0.9    &0.3    &60 &3.4   &1.551 &2.159 &6.7   &7.3   &$-$40.272 \nl
4 & 9607.4 &$-$0.4 &$-$0.4 &60 &1.4   &1.574 &2.343 &8.0   &8.9   &$-$33.417 \nl
5 &31638.6 &1.6    &0.1    &39 &315.2 &3.075 &7.488 &92.1  &25.3  &$-$12.052 \nl
\tableline
\end{tabular}
\end{center}

% Text for table footnotes must follow the tabular environment but must
% be inside the table environment.  Note that it is OK to put \ref's
% in \tablenotetext's.

\tablenotetext{a}{Sample footnote for Table~\ref{O4.1-tbl-2}}
\tablenotetext{b}{Another sample footnote for Table~\ref{O4.1-tbl-2}}
\tablecomments{Table end notes apply to the entire table}
\end{table}
%
However, reducing the text size does sacrifice the readability of the
numbers in the table, and the table environment does allow more
control over the spacing of the columns.  
In each table, note the use of the math mode minus
sign (\texttt{\$-\$}) instead of the text mode dash.

% Finally, we have a little acknowledgments section. Do NOT place 
% acknowledgments in a separate \section. 

\acknowledgments

We are grateful to V.\ Barger, T.\ Han, and R.~J.~N.\ Phillips for doing
the math in section \ref{hairymath} of this paper.

%-----------------------------------------------------------------------
%			      References
%-----------------------------------------------------------------------
% List your references below within the reference environment
% (i.e. between the \begin{references} and \end{references} tags).
% Each new reference should begin with a \reference command which sets
% up the proper indentation.  
%    NOTE: all citations in the text _must_ have a corresponding entry in 
%    the reference list, and all references must be cited in the text.
%
% Observe the following order when listing bibliographical 
% information for each reference:  author name(s), publication 
% year, journal name, volume, and page number for articles. 
% URLs to the reference may be given either in-line, or as a footnote. 
% Note that many journal names are available as macros; see
% the User Guide for a listing "macro-ized" journals. 
%
% You may find the following tricks to be helpful:
%
%   o  "\ " after a period prevents LaTeX from interpreting the period 
%      as an end of a sentence.
%   o  \adassxvi is a macro that expands to the full title, editor,
%      and publishing information for the ADASS XVI conference
%      proceedings.  Such macros are defined for ADASS conferences I
%      through the most recently published proceedings.
%   o  When referencing a paper in the current volume, use the
%      \adassxvii and \paperref macros.  The argument to \paperref is
%      the paper ID code for the paper you are referencing.  See the 
%      note in the "Paper ID Code" section above for details on how to 
%      determine the paper ID code for the paper you reference.  

\begin{references}

\reference Auri\`ere, M.\  1982, \aap, 109, 301
\reference Djorgovski, S., \& King, I.\ R.\  1984, \apj, 277, L49
\reference Hagiwara, K., \& Zeppenfeld, D.\  1986, Nucl.Phys., 274, 1
\reference H\'enon, M.\  1961, Ann.d'Ap., 24, 369
\reference King, I.~R.\  1966, \aj, 71, 276
\reference King, I.~R.\  1975, in Dynamics of Stellar Systems, ed.\ A.\ Hayli
    (Dordrecht: Reidel), 99
\reference King, I.\ R., Hedemann, E., Hodge, S\ M., \& White, R.~E.
    1968, \aj, 73, 456
\reference Kron, G.~E., Hewitt, A.~V., \& Wasserman, L.~H.\
    1984, \pasp, 96, 198
\reference Lynden-Bell, D., \& Wood, R.\  1968, \mnras, 138, 495
\reference Ortolani, S., Rosino, L., \& Sandage, A.\  1985, \aj, 90, 473
\reference Peterson, C.~J.\  1976, \aj, 81, 617
\reference Roberts, D.~A.\ 2008, \adassxvii, \paperref{P1.3}
\reference Spitzer, L.\  1985, in Dynamics of Star Clusters,
    ed.~J.~Goodman \& P.~Hut (Dordrecht: Reidel), 109
    
\end{references}

% Do not place any material after the references section

\end{document}  % Leave intact
